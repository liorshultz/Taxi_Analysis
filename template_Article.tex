\documentclass[]{article}


\newcommand{\rand}{ $ RAND \frac{1}{2} $ }

%opening
\title{A Tale of Two Cities}
\author{Prof. Yishay Mansour, Mr. Lior Shultz}




\begin{document}

\maketitle

\begin{abstract}

We consider the following problem to achieve a basic understanding of taxi driver on-line decisions:
	We have two cities and jobs arrive one after another and the taxi driver can decide for each job if he wants to pick up the passenger or forfeit the job.
	Accepting a job yields the driver a reward of $ 1 $. However, accepting a job in a different city means that the driver will have to drive to the other city and pay $ 1 $ for the relocation - effectively earning nothing for the fare.

\end{abstract}

\section{Notation}

We will denote the following:

1. $ T $ - The number of total jobs received.

2. $ L , R $ - The two cities.

\section{Deterministic Algorithm}

The simple problem of a deterministic algorithm is largely uninteresting in this case since we can at each point in the algorithm a job can hail from the other city meaning that the $ ALG $ will receive $ 0 $ profit. $ OPT $ however will get at the very least a profit of $ \frac{T}{2} $ at least by simply choosing the city that has more jobs in it and staying there. This means that any deterministic algorithm can't achieve any finite competitive ratio.

\section{Online simple strategies}

We will discuss the following two simple strategies:

1. $ STAY $ - At the beginning we will use a coin to decide on a city and simply stay there for the rest of the algorithm. It is easy to see that this algorithm is expected to earn $ \frac{T}{2} $.

2. \rand - Every turn of the algorithm if the job is in a different city we will flip a coin and move to that city with probability $ \frac{1}{2} $. If the job is in the city we are already in we will of course accept it along with our reward.

\section{Rand Analysis}

Let's look at a possible way of analyzing the profit of the \rand algorithm. At time $ t $ we have a probability of $ p $ of being in $ R $ at the beginning of the turn and probability of $ 1 - p $ of being in $ L $ at the beginning of the turn. This means that we can expect to earn $ p $ and update the probability of being at $ R $ at time $ t + 1 $ to $ p + \frac{1}{2} \dot ( 1 - p ) = \frac{1}{2} + \frac{1}{2}p$ and the probability of being at $ L $ to $ \frac{1}{2} \dot ( 1 - p ) $. In the case of consistently alternating jobs (the job sequence consisting of $ L, R, L, R, L, \ldots $) we can see that a balance is achieved when $ p = \frac{1}{3} $. This can be easily demonstrated by the fact that through simple calculation we can see that the probabilities are swapped after the job arrives at $ R $ and thus they can continuously swap and remain balanced that way. This means that in every step we will earn exactly $ \frac{1}{3} $ achieving a total expected revenue of $ \frac{T}{3} $ compared to the $ \frac{T}{2} $ of $ OPT $.

\end{document}
